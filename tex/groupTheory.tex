\section{群论基础}

\subsection{}

\noindent 集合+元素间的二元运算
\begin{itemize}
    \item 封闭性
    \item 满足结合律
    \item 存在幺元
    \item 存在逆元
\end{itemize}

\subsection{左移、右移和划分}

$a\in G$,群的左移$aG$、右移$Ga$。

群$H$是群$G$的一个子群,$H$的左移$aH(a\in G)$构成一个群元素的集合。

如果$b\in aH$,
我们可以设$b= ah_{i},\,h_{i}\in H$,
%h_{i}\in H\,and\,H\,is\,a\,group\rightarrow h_{i}^{-1}\in H\\
则$bH=ah_{i}H$,
由于$H$是一个群,所以集合$H$等于集合$h_{i}H$,
所以$bH=a(h_{i}H)=aH$。

等价关系$b\in aH\rightleftharpoons a\in bH$可以用于将G的元素划分为互斥的多个集合
$\{g_{1}H,g_{2}H...g_{n}H\ |\ g_{i}H\cap g_{j}H=\emptyset\}$,
且$g_{1}H\cup g_{2}H\cup ...\cup g_{n}H=G$。

右移$Ha$同理。

%%%%%%%%%% 错误的 %%%%%%%%%%
%对于$aHa^{-1}$,同样的推理过程
%\begin{gather}
%    b\in aHa^{-1}\\
%    b=ah_{i}a^{-1},\ h_{i}\in H\\
%    bHb^{-1}=ah_{i}a^{-1}Hah_{i}^{-1}a^{-1}\\
%    if\ H=a^{-1}Ha,\rightarrow
%    \ bHb^{-1}=ah_{i}(a^{-1}Ha)h_{i}^{-1}a^{-1}=ah_{i}Hh_{i}^{-1}a^{-1}=aHa^{-1}
%\end{gather}
%如果对于$G$的任意群元素$a$有$H=aHa^{-1}$,
%则$b\in aHa^{-1}\rightleftharpoons a\in bHb^{-1}$构成一个等价关系。
%%%%%%%%%%%%%%%%%%%%%%%%%%%%

\subsection{群同态和商群}

$\forall\ a\in G:\ aH=_{S}Ha\ (or\ H=_{S}aHa^{-1})$
\footnote{$=_{S}$表示集合相等,$=_{G}$表示群同构,$H \leq_{G} G$表示群$H$是群$G$的子群},
称$H$为$G$的一个正规子群。

$aHbH=_{S}abHH=_{S}abH$,正规子群的左移之间的运算结果仍然是正规子群的一个左移(封闭性)。

$aHH=_{S}aH,\ HaH=_{S}aHH=_{S}aH$,正规子群与其任一个左移之间的运算得到该左移自身(存在幺元)。

$aHa^{-1}H=_{S}aa^{-1}HH=_{S}HH=_{S}H$,群元素$a$及其逆元$a^{-1}$作用于正规子群$H$得到的左移之间的运算得到正规子群$H$自身(存在逆元)。
\\

群映射$\phi:G\rightarrow H$满足$\phi(ab)=\phi(a)\phi(b)\ (a,b\in G)$

$Ker\ \phi=\{a\ |\ a\in G,\ \phi(a)h=h\ (\forall h\in H)\}$

$Im\ \phi=\{h=\phi(a)\ |\ \forall a\in G\}\subseteq H$


