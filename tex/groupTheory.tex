\section{群论基础}

\subsection{}

\noindent 集合+元素间的二元运算
\begin{itemize}
    \item 封闭性
    \item 满足结合律
    \item 存在左幺元
    \item 存在左逆元
\end{itemize}

$b$是$a$的左逆元,$ba=1$。
问$ab=?$
看看定义中有什么规则可以利用,看到第三项,$ab=e_{L}ab$。
然后呢?看到定义第四项,有$\forall\ g\in G:\ e_{L}=g^{-1}g$。
$g$取什么好?看到$e_{L}ab$中有关系的有$a$、$b$,尝试$a^{-1}a$和$b^{-1}b$,
发现$b^{-1}bab=b^{-1}(ba)b=b^{-1}e_{L}b=b^{-1}b=e_{L}$,
所以,$ab=e_{L}$,$b$也是$a$的右幺元。

\subsection{左移、右移和划分}

$a\in G$,群的左移$aG$、右移$Ga$。

群$H$是群$G$的一个子群,$H$的左移$aH(a\in G)$构成一个群元素的集合。

如果$b\in aH$,
我们可以设$b= ah_{i},\,h_{i}\in H$,
%h_{i}\in H\,and\,H\,is\,a\,group\rightarrow h_{i}^{-1}\in H\\
则$bH=ah_{i}H$,
由于$H$是一个群,所以集合$H$等于集合$h_{i}H$,
所以$bH=a(h_{i}H)=aH$。

$\exists g\in G:g\in aH,\ g\in bH\rightarrow gH=_{S}aH,\ gH=_{S}bH\rightarrow aH=_{S}bH$,
所以,$aH\neq_{S} bH\rightarrow \centernot\exists g\in G:g\in aH,\ g\in bH\rightarrow aH\cap bH=\emptyset$

等价关系$b\in aH\rightleftharpoons a\in bH$可以用于将G的元素划分为互斥的多个集合
$\{g_{1}H,g_{2}H...g_{n}H\ |\ g_{i}H\cap g_{j}H=\emptyset\}$,
且$g_{1}H\cup g_{2}H\cup ...\cup g_{n}H=G$。

右移$Ha$同理。

%%%%%%%%%% 错误的 %%%%%%%%%%
%对于$aHa^{-1}$,同样的推理过程
%\begin{gather}
%    b\in aHa^{-1}\\
%    b=ah_{i}a^{-1},\ h_{i}\in H\\
%    bHb^{-1}=ah_{i}a^{-1}Hah_{i}^{-1}a^{-1}\\
%    if\ H=a^{-1}Ha,\rightarrow
%    \ bHb^{-1}=ah_{i}(a^{-1}Ha)h_{i}^{-1}a^{-1}=ah_{i}Hh_{i}^{-1}a^{-1}=aHa^{-1}
%\end{gather}
%如果对于$G$的任意群元素$a$有$H=aHa^{-1}$,
%则$b\in aHa^{-1}\rightleftharpoons a\in bHb^{-1}$构成一个等价关系。
%%%%%%%%%%%%%%%%%%%%%%%%%%%%

\subsection{群同态和商群}

$\forall\ a\in G:\ aH=_{S}Ha\ (or\ H=_{S}aHa^{-1})$
\footnote{$=_{S}$表示集合相等,$=_{G}$表示群同构,$H \leq_{G} G$表示群$H$是群$G$的子群},
称$H$为$G$的一个正规子群。

$aHbH=_{S}abHH=_{S}abH$,正规子群的左移之间的运算结果仍然是正规子群的一个左移(封闭性)。

$aHH=_{S}aH,\ HaH=_{S}aHH=_{S}aH$,正规子群与其任一个左移之间的运算得到该左移自身(存在幺元)。

$aHa^{-1}H=_{S}aa^{-1}HH=_{S}HH=_{S}H$,群元素$a$及其逆元$a^{-1}$作用于正规子群$H$得到的左移之间的运算得到正规子群$H$自身(存在逆元)。
\\

反之,如果$H$是$G$的一个子群,满足$\forall\ a,b\in G:aHbH=_{S}abH$,
则$aHa^{-1}H=_{S}aa^{-1}H=_{S}H
\rightarrow \forall\ h\in H:aha^{-1}=ah(\in aH)a^{-1}e_{H}(\in a^{-1}H)\in aHa^{-1}H=_{S}H
\rightarrow aHa^{-1}=_{S}H$。
\\

群映射$\phi:G\rightarrow H$满足$\phi(ab)=\phi(a)\phi(b)\ (a,b\in G)$
\footnote{此处,符号$H$不表示$G$的正规子群}

$Ker\ \phi=\{a\ |\ a\in G,\ \phi(a)h=h\ (\forall h\in H)\}$

$Im\ \phi=\{h=\phi(a)\ |\ \forall a\in G\}\subseteq H$
\\

$e_{H}$是群$H$的幺元,$\forall\ k\in\ Ker\ \phi:\ \phi(k)=e_{H}$。

$\forall\ k_{1},k_{2}\in Ker\ \phi:\ \phi(k_{1}k_{2})=\phi(k_{1})\phi(k_{2})=e_{H}e_{H}=e_{H}
\rightarrow k_{1}k_{2}\in Ker\ \phi$,
所以,$K=Ker\ \phi$是$G$的一个子群。
(存在幺元:$e_{G}$是群$G$的幺元,
$\forall\ a\in G:\ \phi(e_{G})\phi(a)=\phi(e_{G}a)=\phi(a)\rightarrow \phi(e_{G})=e_{H}$;
存在逆元:$\forall\ k\in K:\ \phi(kk^{-1})=\phi(k)\phi(k^{-1})=e_{H}\phi(k^{-1}),
\ \phi(kk^{-1})=\phi(e_{G})=\ e_{H}
\rightarrow \phi(k^{-1})=e_{H}\rightarrow k^{-1}\in K$)

$A=\{a\ |\ \phi(a)=h_{A}, a\in G\}$,
$\forall\ a\in A,\ \forall\ k\in K:\ \phi(ak)=\phi(a)\phi(k)=h_{A}e_{H}=h_{A}
\rightarrow ak\in A
\rightarrow aK\subseteq A$
($\forall a,b\in A:\phi(a)=\phi(b),
\ \phi(a^{-1}b)=\phi(a^{-1})\phi(b)=\phi(a)^{-1}\phi(b)=\phi(a)^{-1}\phi(a)=e_{H}
\rightarrow a^{-1}b\in K
\rightarrow a^{-1}A\subseteq K$,
综上$aK\subseteq A$、$a^{-1}A\subseteq K$可知$|A|=|K|$,所以$aK=_{S}A$)

$A^{-1}=\{a^{-1}\ |\ \phi(a^{-1})=h_{A^{-1}}=h_{A}^{-1}, a^{-1}\in G\}$,
$\forall\ a\in A,\ \forall\ a^{-1}\in A^{-1},\ \forall\ k\in K:\ \phi(aka^{-1})=\phi(a)\phi(k)\phi(a^{-1})=h_{A}e_{H}h_{A}^{-1}=e_{H}
\rightarrow aka^{-1}\in K
\rightarrow aKa^{-1}=_{S} K$
($=_{S}$号成立,因为集合$aKa^{-1}$元素的个数等于集合$K$元素的个数)
\\

$K$是群$G$的一个正规子群,所有陪集的集合$\{gK\}\ (g\in G)$构成$G$的一个划分,
$a,g\in G,\ \phi(g)=\phi(a)\ iff\ g\in aK$,
证明如下:
\\
1)$g\in aK\rightarrow \exists k\in K:g=ak,\ \phi(g)=\phi(ak)=\phi(a)\phi(k)=\phi(a)e_{H}=\phi(a)$
\\
2)$g\notin aK\rightarrow a^{-1}g\notin K\rightarrow \phi(a^{-1}g)\neq e_{H}
\rightarrow \phi(g)=\phi(aa^{-1}g)=\phi(a)\phi(a^{-1}g)\neq \phi(a)e_{H}=\phi(a)$

