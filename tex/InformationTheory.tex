\section{}
\subsection{Jensen不等式与K-L散度}

\begin{gather}
    for \quad x_{1} < x_{2} < ... < x_{n}\\
    a_{1}x_{1}+a_{2}x_{2}+...+a_{n}x_{n}\\
    = 1\cdot x_{1}+(1-a_{1})\cdot(x_{2}-x_{1})+(1-a_{1}-a_{2})\cdot(x_{3}-x_{2})+\\
    ...+(1-a_{1}-a_{2}-...-a_{n-1})\cdot(x_{n}-x_{n-1})\label{eq:1}\\
    \intertext{\hspace*{\fill}(其中,$a_{i}>0$且$a_{1}+a_{2}+...+a_{n-1}+a_{n}=1$)\hspace*{\fill}}\\
    c_{1}=1,\, c_{2}=1-a_{1},\, c_{3}=1-a_{1}-a_{2}\, ...\, c_{n}=1-a_{1}-a_{2}-...-a_{n-1}\\
\end{gather}

对于一个凸函数$f(x)$,用直线段$L_{i+1}$连接$(x_{i},f(x_{i}))$和$(x_{i+1},f(x_{i+1}))$,
$\Delta x_{i} = x_{i}-x_{i-1}$和$\Delta f_{i} = f(x_{i})-f(x_{i-1})$是$L_{i}$在x-轴和y-轴的投影长度。

将直线段$\{c_{2}L_{2},\,c_{3}L_{3},\,...\,,\,c_{n}L_{n}\}$按顺序首尾相接得到折线$L$,
整体平移$L$使得$c_{2}L_{2}$端的起始点为$(x_{1},f(x_{1}))$,
此时$c_{n}L_{n}$端的终点的x坐标即为式\ref{eq:1}的值,
y坐标的值为:
\begin{gather}
    a_{1}f(x_{1})+a_{2}f(x_{2})+...+a_{n}f(x_{n})
\end{gather}

由图可知,$\{c_{2}L_{2},\,c_{3}L_{3},\,...\,,\,c_{n}L_{n}\}$整体永远位于$\{L_{2},\,L_{3},\,...\,,\,L_{n}\}$上方,
所以,
\begin{align}
    y(\{c_{2}L_{2},\,c_{3}L_{3},\,...\,,\,c_{n}L_{n}\})
    > y(\{L_{2},\,L_{3},\,...\,,\,L_{n}\})
    \geq f(x)
\end{align}
即,
\begin{align}
    a_{1}f(x_{1})+a_{2}f(x_{2})+...+a_{n}f(x_{n})
    \geq f(a_{1}x_{1}+a_{2}x_{2}+...+a_{n}x_{n})\quad (\text{Jensen不等式})
\end{align}

由Jensen不等式可知,K-L散度
\begin{align}
    &D_{KL}(P||Q)\\
    =&(\sum_{i}-P_{i}ln(Q_{i}))-(\sum_{i}-P_{i}ln(P_{i}))\\
    =&\sum_{i}-P_{i}ln(\frac{Q_{i}}{P_{i}})\\
    \geq&-ln(\sum_{i}P_{i}\frac{Q_{i}}{P_{i}})\\
    =&-ln(\sum_{i}Q_{i}) = 0\\
    \intertext{($f(x)=-ln(x)$是凸函数,$P_{i}>0$,$Q_{i}>0$,$\sum_{i}P_{i}=1$,$\sum_{i}Q_{i}=1$)}
\end{align}