\section{置换群和行列式}

有限群G的任意一个群元素g作用于该群所有的群元素,相当于在所有群元素之间进行了一次置换操作
$\rightarrow$所有的有限群都同构于置换群
(孟道骥书\cite{meng2010}p26-27)

置换群的一个群元素可分割为多个独立的轮换操作,每个轮换又可拆解为一步一步的对换操作

根据群元素包含对换操作的次数可定义其正负号,$(-1)^{\text{对换次数}}$

$n$个对象之间的置换操作有$n!$种。

行列式$|A_{ij}|$展开式的某一项$A_{1p(1)}A_{2p(2)}...A_{np(n)}$,$p$代表置换操作,遍历所有可能的置换操作:
\begin{gather}
    det(A)=\sum_{i}^{n!} (-1)^{p_{i}}A_{1p_{i}(1)}A_{2p_{i}(2)}...A_{np_{i}(n)}
\end{gather}
\\

几何上,平行多面体的体积
\begin{align}
    &\left | \left | \begin{matrix}
        x_{1} &y_{1} & z_{1} \\
        x_{2} &y_{2} & z_{2} \\
        x_{3} &y_{3} & z_{3}
    \end{matrix} \right | \right | \\
    =& \left | x_{1}\left | \begin{matrix} y_{2} & z_{2} \\ y_{3} & z_{3} \end{matrix} \right |
    - y_{1}\left | \begin{matrix} x_{2} & z_{2} \\ x_{3} & z_{3} \end{matrix} \right |
    + z_{1}\left | \begin{matrix} x_{2} & y_{2} \\ x_{3} & y_{3} \end{matrix} \right | \right |
\end{align}
其中,$y_{1}$项前面的符号是因为$\left | \begin{matrix} x_{2} & z_{2} \\ x_{3} & z_{3} \end{matrix} \right |$中,$x$坐标和$z$坐标的顺序反了。
由此可以推广到高维情况下,行列式展开的正负号也是坐标排列的顺序问题导致的——在规定的左手(右手)坐标系中,只有正确的坐标顺序才能正确计算平行多面体的体积。

$(x_{i}, y_{i}, z_{i})$代表平行多面体某条边的向量,没有理由规定它们在行列式中以某种排列顺序出现,因此外加绝对值号
