\section{笔记}

\subsection{}
交换任意两行,行列式反号
$\rightarrow$交换任意两列
$\rightarrow$计算行列式某一项时,通过交换行,重新排列,使该项的所有元素都排列在对角线上,以确定该项的正负号
$\rightarrow$转置矩阵则通过交换列重新排列使某一项的所有元素都排列在对角线上,由于转置后交换列相当于转置前交换行,
至此证明了对于行列任一相同的项,转置前后其符号相同,即矩阵与其转置矩阵的行列式相等。

\subsection{矩阵相乘的行列式}

\begin{gather}
    for\,
    \begin{pmatrix}
        a_{i1} & a_{i2} & ... & a_{in}
    \end{pmatrix}
    =\begin{pmatrix}
        b_{i1} & b_{i2} & ... & b_{in}
    \end{pmatrix}
    +\begin{pmatrix}
        c_{i1} & c_{i2} & ... & c_{in}
    \end{pmatrix}\\
    =\begin{pmatrix}
        b_{i1}+c_{i1} & b_{i2}+c_{i2} & ... & b_{in}+c_{in}
    \end{pmatrix}\\
    \left |
    \begin{matrix}
        a_{11} & a_{12} & ... & a_{1n}\\
        a_{21} & a_{22} & ... & a_{2n}\\
        \vdots & \vdots & ... & \vdots\\
        a_{i1} & a_{i2} & ... & a_{in}\\
        \vdots & \vdots & ... & \vdots\\
        a_{n1} & a_{n2} & ... & a_{nn}\\
    \end{matrix}
    \right |
    =    \left |
    \begin{matrix}
        a_{11} & a_{12} & ... & a_{1n}\\
        a_{21} & a_{22} & ... & a_{2n}\\
        \vdots & \vdots & ... & \vdots\\
        b_{i1} & b_{i2} & ... & _{in}\\
        \vdots & \vdots & ... & \vdots\\
        a_{n1} & a_{n2} & ... & a_{nn}\\
    \end{matrix}
    \right |
    +    \left |
    \begin{matrix}
        a_{11} & a_{12} & ... & a_{1n}\\
        a_{21} & a_{22} & ... & a_{2n}\\
        \vdots & \vdots & ... & \vdots\\
        c_{i1} & c_{i2} & ... & c_{in}\\
        \vdots & \vdots & ... & \vdots\\
        a_{n1} & a_{n2} & ... & a_{nn}\\
    \end{matrix}
    \right |
    \label{eq:detAdd}
\end{gather}

\begin{gather}
    \left | AB \right |
    =\left | \begin{matrix}
    \begin{matrix}
        \quad A_{11}\times\begin{pmatrix}
            B_{11} & B_{12} & ... & B_{1n}
        \end{pmatrix}\\
        +A_{12}\times\begin{pmatrix}
            B_{21} & B_{22} & ... & B_{2n}
        \end{pmatrix}\\
        \vdots\\
        +A_{1n}\times\begin{pmatrix}
            B_{n1} & B_{n2} & ... & B_{nn}
        \end{pmatrix}\\
    \end{matrix}\\
    \\
    \begin{matrix}
        \quad A_{21}\times\begin{pmatrix}
            B_{11} & B_{12} & ... & B_{1n}
        \end{pmatrix}\\
        +A_{22}\times\begin{pmatrix}
            B_{21} & B_{22} & ... & B_{2n}
        \end{pmatrix}\\
        \vdots\\
        +A_{2n}\times\begin{pmatrix}
            B_{n1} & B_{n2} & ... & B_{nn}
        \end{pmatrix}\\
    \end{matrix}\\
    \\
    \vdots\\
    \\
    \begin{matrix}
        \quad A_{n1}\times\begin{pmatrix}
            B_{11} & B_{12} & ... & B_{1n}
        \end{pmatrix}\\
        +A_{n2}\times\begin{pmatrix}
            B_{21} & B_{22} & ... & B_{2n}
        \end{pmatrix}\\
        \vdots\\
        +A_{nn}\times\begin{pmatrix}
            B_{n1} & B_{n2} & ... & B_{nn}
        \end{pmatrix}\\
    \end{matrix}\\
    \end{matrix} \right |
    \label{eq:ABdetExpansion}
\end{gather}
由于交换矩阵的任意两行,行列式反号,如果矩阵中有任意两行完全相同,则其行列式为零。
将式\ref{eq:ABdetExpansion}按式\ref{eq:detAdd}展开,只有如下形式的项不为零:
\begin{gather}
    \left | \begin{matrix}
    A_{1p(1)}\times\begin{pmatrix}
        B_{p(1)1} & B_{p(1)2} & ... & B_{p(1)n}\\
    \end{pmatrix}\\
    A_{2p(2)}\times\begin{pmatrix}
        B_{p(2)1} & B_{p(2)2} & ... & B_{p(2)n}\\
    \end{pmatrix}\\
    \vdots\\
    A_{np(n)}\times\begin{pmatrix}
        B_{p(n)1} & B_{p(n)2} & ... & B_{p(n)n}\\
    \end{pmatrix}\\
    \end{matrix} \right |
    \label{eq:ABdetExpansion2}
\end{gather}
其中$p$是$n$个对象的一个置换变换。
交换式\ref{eq:ABdetExpansion2}各行的顺序变为:
\begin{gather}
    \left | \begin{matrix}
        A_{p^{-1}(1)1}\times\begin{pmatrix}
            B_{11} & B_{12} & ... & B_{1n}\\
        \end{pmatrix}\\
        A_{p^{-1}(2)2}\times\begin{pmatrix}
            B_{21} & B_{22} & ... & B_{2n}\\
        \end{pmatrix}\\
        \vdots\\
        A_{p^{-1}(n)n}\times\begin{pmatrix}
            B_{n1} & B_{n2} & ... & B_{nn}\\
        \end{pmatrix}\\
    \end{matrix} \right |
\end{gather}
交换前后行列式的值相差一个正负号$(-1)^{p}$。
式\ref{eq:ABdetExpansion}所有不为零的展开项相加得到:
\begin{align}
    &\sum_{i=1}^{n!} (-1)^{p_{i}}
    \left | \begin{matrix}
        A_{p_{i}^{-1}(1)1}\times\begin{pmatrix}
            B_{11} & B_{12} & ... & B_{1n}\\
        \end{pmatrix}\\
        A_{p_{i}^{-1}(2)2}\times\begin{pmatrix}
            B_{21} & B_{22} & ... & B_{2n}\\
        \end{pmatrix}\\
        \vdots\\
        A_{p_{i}^{-1}(n)n}\times\begin{pmatrix}
            B_{n1} & B_{n2} & ... & B_{nn}\\
        \end{pmatrix}\\
    \end{matrix} \right |\\
    =&\sum_{i=1}^{n!} (-1)^{p_{i}}
    A_{p_{i}^{-1}(1)1}A_{p_{i}^{-1}(2)2}...A_{p_{i}^{-1}(n)n}
    \times \left | \begin{matrix}
        B_{11} & B_{12} & ... & B_{1n}\\
        B_{21} & B_{22} & ... & B_{2n}\\
        \vdots\\
        B_{n1} & B_{n2} & ... & B_{nn}
    \end{matrix} \right |\\
    =&\left | B \right |
    \sum_{i=1}^{n!} (-1)^{p_{i}}
    A_{p_{i}^{-1}(1)1}A_{p_{i}^{-1}(2)2}...A_{p_{i}^{-1}(n)n}\\
    =&\left | B \right |
    \sum_{i=1}^{n!} (-1)^{p_{i}}
    A_{1p_{i}(1)}A_{2p_{i}(2)}...A_{np_{i}(n)}\\
    =&\left | B \right |
    \left | A \right |
\end{align}

\subsection{}
\begin{gather}
    det(A-\lambda I)=0
    \rightarrow (\lambda - \lambda_{1})(\lambda - \lambda_{2})...(\lambda - \lambda_{N})=0
\end{gather}
(1) $\prod_{i}^{N} \lambda_{i}=det(A)$有两种证法:\\
(1.1)  $\prod_{i}^{N}\lambda_{i}$即行列式$det(A-\lambda I)$展开的常数项,
行列式$det(A-\lambda I)$和行列式$det(A)$均按形式展开,两者展开式中不包含对角元素的项(part1)完全相同,
$det(A-\lambda I)$包含一个或多个对角元素$(A_{ii}-\lambda)$的项与$det(A)$包含一个或多个对角元素$A_{ii}$的项形式一致,
把包含$(A_{ii}-\lambda)$的项的括号展开,分为包含$\lambda$(part2)与不包含$\lambda$(part3)两部分,
则不包含$\lambda$的部分形式与$det(A)$包含对角元素$A_{ii}$的项完全相同,
至此$det(A-\lambda I)$的形式展开项被分为part1、part2、part3三部分,其中不包含$\lambda$的part1和part3合起来等于$det(A)$,
剩下的part2是包含至少一个$\lambda$的项。\\
(1.2)
\begin{gather}
    U^{-1}AU
    =\begin{pmatrix}
        \lambda_{1} &  &  & \\
         & \lambda_{2} &  & \\
         &  & \ddots & \\
         &  &  & \lambda_{N}
    \end{pmatrix}\\
    det(AB)=det(A)det(B)\\
    det(A)=det(diag(\lambda_{1}, \lambda_{2}, ..., \lambda_{N}))=\prod_{i}^{N} \lambda_{i}
\end{gather}
(2) $\sum_{i}^{N} \lambda_{i}=trace(A)$
\begin{gather}
    trace(A)=\sum_{i}A_{ii}\\
    C=AB, C_{ij}=\sum_{k}A_{ik}B_{kj}\\
    trace(AB)=trace(C)=\sum_{i}C_{ii}=\sum_{i} \sum_{k}A_{ik}B_{ki}\\
    trace(BA)=\sum_{i} (BA)_{ii} = \sum_{i} \sum_{k} B_{ik}A_{ki}\\
    trace(AB)=trace(BA)
\end{gather}
\begin{align}
    &trace(A)\\
    =&trace(U
    \begin{pmatrix}
        \lambda_{1} &  &  & \\
         & \lambda_{2} &  & \\
         &  & \ddots & \\
         &  &  & \lambda_{N}
    \end{pmatrix}
    U^{-1})\\
    =&trace(U^{-1}U
    \begin{pmatrix}
        \lambda_{1} &  &  & \\
         & \lambda_{2} &  & \\
         &  & \ddots & \\
         &  &  & \lambda_{N}
    \end{pmatrix}
    )\\
    =&trace(
    \begin{pmatrix}
        \lambda_{1} &  &  & \\
         & \lambda_{2} &  & \\
         &  & \ddots & \\
         &  &  & \lambda_{N}
    \end{pmatrix}    
    )\\
    =&\sum_{i}^{N} \lambda_{i}
\end{align}

\subsection{雅可比行列式}

坐标系变换
\begin{gather}
    (x,y)\rightarrow (\mu(x,y),\nu(x,y))
\end{gather}
新坐标系下微元$d\mu d\nu$表示向量$(d\mu \bm{\overrightarrow{\mu}},0\bm{\overrightarrow{\nu}})$和$(0\bm{\overrightarrow{\mu}},d\nu \bm{\overrightarrow{\nu}})$围成的长方形的面积,
对应到原坐标系下的微元区域为向量$(\frac{\partial x}{\partial \mu}d\mu\bm{\overrightarrow{x}},\frac{\partial y}{\partial \mu}d\mu\bm{\overrightarrow{y}})$
和$(\frac{\partial x}{\partial \nu}d\nu\bm{\overrightarrow{x}},\frac{\partial y}{\partial \nu}d\nu\bm{\overrightarrow{y}})$围成的平行四边形,
其面积为:
\begin{gather}
    \left | \left |
    \begin{matrix}
        \frac{\partial x}{\partial \mu}d\mu & \frac{\partial y}{\partial \mu}d\mu\\
        \frac{\partial x}{\partial \nu}d\nu & \frac{\partial y}{\partial \nu}d\nu
    \end{matrix}
    \right | \right |
    =\left | \left |
    \begin{matrix}
        \frac{\partial x}{\partial \mu} & \frac{\partial y}{\partial \mu}\\
        \frac{\partial x}{\partial \nu} & \frac{\partial y}{\partial \nu}
    \end{matrix}
    \right | \right | d\mu d\nu
\end{gather}

$f(x,y)$在微元$dxdy$区域内的积分为$f(x,y)dxdy$,
在微元$d\mu d\nu$对应的原$x-y$坐标系下的区域的积分为
\begin{gather}
    f(x,y)
    \left | \left |
    \begin{matrix}
        \frac{\partial x}{\partial \mu} & \frac{\partial y}{\partial \mu}\\
        \frac{\partial x}{\partial \nu} & \frac{\partial y}{\partial \nu}
    \end{matrix}
    \right | \right | d\mu d\nu
    =f(x(\mu,\nu),y(\mu,\nu))
    \left | \left |
    \begin{matrix}
        \frac{\partial x}{\partial \mu} & \frac{\partial y}{\partial \mu}\\
        \frac{\partial x}{\partial \nu} & \frac{\partial y}{\partial \nu}
    \end{matrix}
    \right | \right | d\mu d\nu
    =g(\mu,\nu)
    \left | \left |
    \begin{matrix}
        \frac{\partial x}{\partial \mu} & \frac{\partial y}{\partial \mu}\\
        \frac{\partial x}{\partial \nu} & \frac{\partial y}{\partial \nu}
    \end{matrix}
    \right | \right | d\mu d\nu
\end{gather}
