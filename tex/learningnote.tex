\section{学习笔记}

\noindent\textbf{孟道骥书\cite{meng2010}theo4.2.2p123}\\

\noindent $x$的轨道$O_{x}=\{y\ |\ y=g(x),\ \forall g\in G\}$\\
$x$的迷向子群$Fx=\{g\ |\ g\in G,\ g(x)=x\}$\\

\noindent$\forall\ y \in O_{x}:\exists g\in G:y=g(x)$\\
$\rightarrow \forall\ f\in F_{x}:gfg^{-1}(y)=gf(x)=g(x)=y$\\
$\rightarrow gF_{x}g^{-1}\subseteq F_{y}$\\

\noindent 如果$\exists g\in G:g\neq e_{G},\ \forall x\in X:g(x)=x$\\
则$g\in \cap_{i=1}^{m} F_{x_{i}}= F_{x_{1}}\cap F_{x_{2}}...\cap F_{x_{m}} \ (x_{i}\in X,\ m=|X|)$\\
由于$F_{x_{i}}$是群,所以$\cap_{i=1}^{|X|} F_{x_{i}}$也是群\\

\noindent 要证明theo4.2.2(2):$G$在$O_{x}$上作用有效$\rightleftharpoons$$F_{x}$中所包含的$G$的正规子群仅为$\{e\}$,只需证明:\\
1)$F_{x}$包含的$G$的正规子群大于${e}$$\rightarrow$$G$在$O_{x}$上作用不是有效的\\
2)$F_{x}$包含的$G$的正规子群大于${e}$$\leftarrow$$G$在$O_{x}$上作用不是有效的\\

\noindent 子群$H\subseteq F_{x},\ \forall a\in G:aHa^{-1}=_{S}H$\\
$\rightarrow \forall y\in O_{x}:\quad \forall h\in H,\ \exists g\in G:y=g(x),\ ghg^{-1}(y)=gh(x)=g(x)=y$\\
$\rightarrow \forall y\in O_{x}:\quad \forall h\in H,\ \exists g\in G:ghg^{-1}\in F_{y}$\\
$\rightarrow \forall y\in O_{x}:\quad gHg^{-1}\subseteq F_{y}\rightarrow \forall y\in O_{x}:\quad H\subseteq F_{y}$\\
$\rightarrow H\subseteq \cap_{i=1}^{m} F_{y_{i}}\ (y_{i}\in O_{x},\ m=|O_{x}|)$\\
$\rightarrow \forall h\in H:\forall y\in O_{x}:h(y)=y$\\

\noindent $H=\{h\ |\ h\in G,\ \forall y\in O_{x}:h(y)=y\}$\\
a)$\rightarrow H=_{G}\cap_{i=1}^{m} F_{y_{i}}\ (y_{i}\in O_{x},\ m=|O_{x}|)$,$H$是一个群\\
b)$\rightarrow\forall g\in G,\ \forall h\in H:\quad\forall y\in O_{x}:\exists z\in O_{x},\ y=g(z),\ ghg^{-1}(y)=gh(z)=g(z)=y$\\
$\rightarrow\forall g\in G,\ \forall h\in H:\quad ghg^{-1}\in H$\\
$\rightarrow\forall g\in G:gHg^{-1}=_{S}H$\\
a)+b)$\rightarrow H$是一个正规子群\\

\noindent 重新叙述下theo4.2.2(2)的内容:\\
群$G$在集合$O_{x}$上的\textbf{幺作用子群}$H=\{h\ |\ h\in G,\ \forall y\in O_{x}:h(y)=y\}=\cap_{i=1}^{m} F_{y_{i}}\ (y_{i}\in O_{x},\ m=|O_{x}|)$
$\rightarrow \forall\ y_{i}\in O_{x}:H\subseteq F_{y_{i}}$,定理要证明两件事:\\
一、群$G$在集合$O_{x}$上的幺作用子群$H$是$G$的正规子群\\
二、$F_{x}$包含的群$G$的正规子群$H$包含于($\subseteq$)$G$在集合$O_{x}$上的幺作用子群\\

\textbf{孟道骥书\cite{meng2010}theo4.2.3p124}

\begin{gather}
    \forall\ a \in gF_{x}:\exists\ f\in F_{x}:a=gf\\
    \rightarrow \forall\ a \in gF_{x}:a(x)=gf(x)=g(x)
\end{gather}

\textbf{Fraleigh书\cite{Fraleigh2002}10.11Corollary}

一个群的子群的陪集构成群的一个划分真是太神奇了。
考虑一个有限群的一个群元素,它与自身作有限次的二元运算会回到幺元,
因此群的一个群元素与自身不断作二元运算得到群的一个子群,
既然是子群,它的陪集就构成群的一个划分,
因此子群的阶数(也即该群元素与自身作多少次二元运算会回到幺元)是群元素个数的因子。
如果一个有限群包含素数个群元素,则它的子群的包含的群元素的个数只能是1或等于该有限群的阶数,
即该有限群的子群只能是$\{e\}$或它本身,
取该有限群的一个元素不断与自身进行二元运算,得到该群的一个子群,这个子群要么是$\{e\}$,要么是有限群本身,
得到的如果是$\{e\}$,说明取的元素是$e$,
如果取的是$e$以外的其它元素,则得到的子群不可能为$\{e\}$,只能是该有限群自身,
因此,对于一个包含素数个元素的有限群,其非$e$元素与自身不断进行二元运算得到的子群是该有限群自身。\\

取$a$属于有限群$G$,集合$H=\{a,\ a^{2},\ ...,\ a^{k}\}$,
$k$的取值保证$H$的元素之间两两互不相同且$a^{k+1}$必定跟$H$中的某个元素重复,
假设$a^{k+1}=a^{i}\in H$,
若$i=k$则$a=e_{G}\rightarrow k=1$,
若$i=1$,则$a^{k}a=a\rightarrow a^{k}=e_{G}$,
若$2\leq i\leq k-1$,则$a^{k+1}=a^{i}=a^{k}a^{-(k-i)}\rightarrow a=a^{-(k-i)}\rightarrow a^{-1}=a^{k-i}\rightarrow a^{k-i+2}=a\ (3\leq k-i+2\leq k)$,
与$H$的元素之间两两互不相同矛盾。\\

\textbf{孟道骥书\cite{meng2010}theo4.3.1(1)p126}

群$G$对一个集合$X$的作用可以拆分为对多个独立集合(轨道)的作用,每个轨道可以反映$G$不同的子群结构。\\

\textbf{孟道骥书\cite{meng2010}theo4.3.1(2)p126}

$G$(阶数$p^{k}$)对$G$自身元素(阶数$n=p^{k}$)的伴随作用,由theo4.3.1(1)可知,
集合$\{x\ |\ x\in X,\ g(x)=x,\ \forall\ g\in G\}\rightarrow H=\{h\ |\ h\in G,\ ad_{g}(h)=ghg^{-1}=h,\ \forall\ g\in G\}$的个数要么是0,要么是$p$的整数倍。
由于$e_{G}\in H$,因此$|H|$不为0,$|H|$只能是$p$的整数倍。
由$H$的定义知,$h\in H\rightarrow hgh^{-1}=g\ (\forall\ g\in G)\rightarrow H\subseteq C(G)$。
因此,$C(G)\neq \{e_{G}\}$。\\



